\documentclass[12pt]{article}
\usepackage[utf8]{inputenc}
\usepackage{graphicx}
\graphicspath{ {images/} }
\usepackage{authblk}
\bibliographystyle{plain}

\begin{document}
\title{Nest detection and habitat mapping of burrow-nesting seabirds using aerial drone based imagery}

\author{Jiménez López, Jesús[1]; García Perez, Agapito[2]}

\affil[1]{Department of Arcane Biology, University of Miskatonic}
\affil[2]{Department of Astrobiology, University of Lebrija}

\maketitle
\begin{abstract}
	

In the last decade, drones (also known as unmanned aerial systems, remotely piloted aircraft systems, RPAS, UAS, UAV) have been the subject of a growing interest in both the civilian and scientific sphere. Drones offer a relatively risk-free and low-cost manner to rapidly and systematically observe natural phenomena at high spatio-temporal resolution. In this study, a drone was used to capture aerial footage to count North Atlantic White-faced Storm-petrel ground nests and describe habitat-specific environmental conditions.

\end{abstract}

{\bf Keywords:} drones, UAV, remote sensing, wildlife monitoring, conservation, island ecology, Pelagodroma marina, burrow nesting seabirds, Selvagem Grande; Macaronesia, Pitingo

\section{Introduction}

The North Atlantic White-faced Storm-petrel (Pelagodroma marina hypoleuca) digs burrow nests in extremely frangible sandy soils, typically forming dense colonies that are often confined to remote and inhabited islands and islets. Conservation and management efforts require regular monitoring of their nests to assess their breeding status and success, as well as possible impacts of anthropogenic pressures. However, fieldwork and censuses are infrequent and with considerable risk of nest trampling. In this context, drone-based aerial imagery surveys can be a non-intrusive, reliable, and effective method of counting nests where close inspection is impractical or constitutes an unacceptable risk of nest disruption and habitat disturbance \cite{ventura_mapping_2018}. 

In this study, a drone was used to collect aerial photographs to assess the feasibility of detecting and counting nests in the imagery and describe habitat-specific environmental conditions. An ortho-photomosaic of the target nesting area was generated, confirming that in stark contrast to sandy soils and surrounding patches of vegetation, burrows were explicitly visible. Hence, burrows were exhaustively counted by visual inspection and compared with semi-automatic detection methods. In addition, object-based image analysis (OBIA) was applied to the ortho-photomosaic to produce a detailed map of vegetation and landforms. Ecological and environmental conditions driving the spatial distribution of the nests were explored to unravel apparent clustering patterns. We discuss the potential and limitations in the use of drones to detect burrowing nests and mapping breeding areas in remote unpopulated islands, and provide guidelines for improved data acquisition to monitor Important Bird and Biodiversity Area (IBA) over time. 

Our study case illustrates that regular scientific ecological information from conspicuous burrow-nesting seabirds can be acquired using very high-resolution aerial images and different processing analytical tools, with minimal disturbance and reduced risks to the nests integrity, contributing therefore to the conservation of North Atlantic White-faced Storm-petrel.

\section{Methods}


\bibliography{bibliography}

\end{document}